\documentclass[a4paper, 12pt, twoside]{article}

\usepackage[english]{babel}
\usepackage[utf8]{inputenc}
\usepackage[acronym, toc]{glossaries}
\usepackage{comment}
\usepackage{nomencl}
\usepackage{xcolor}
% \usepacakge{minted}

% color
\pagecolor{black}
\color{white}

% glossaries
\makeglossaries

\newacronym{llins}{LLINs}{Long Lasting Insecticide Treated Nets}
\newacronym{abm}{ABM}{Agent-Based Models}
\newacronym{irs}{IRS}{Indoor Residual Spraying}
\newacronym{smc}{SMC}{Seasonal Malaria Chemoprevention}
\newacronym{mcmc}{MCMC}{Markov Chain Monte Carlo}

\newglossaryentry{rts}
{
	name=RTS\,S,
	description={RTS\,S/AS01 is a recombinant protein-based malaria vaccine.}
}

\newglossaryentry{eir}
{
	name=EIR,
	description={Entomological Infectious Rate or infected bites per person per year}
}


% nomenclature
\makenomenclature

\nomenclature{$c$}{Speed of light in a vacuum inertial frame}

% title page
\title{An Introduction on Mathematical Modelling in Malaria Control}
\author{Chunzhe ZHANG}
\date{Jan 2021}

\begin{document}

\begin{titlepage}
\maketitle
\end{titlepage}

\tableofcontents

\section{Introduction}
Mathematical models is a vital way to help public health practitioners to policy decisions. Thus, it is crucial to know how they are made and the assumptions on which they are built. Mathematical models has been introduced into malaria control world for many years. History could go back to 1920s, when Ronald Ross first introduced 
\section{Mathematical Methods}

Types of models:

\subsection{Compartmental Model}

Commonly used in describing disease. 

Deterministic Model 

TODO: add definition

\subsubsection{SIR Model}

\subsubsection{SEIR Model}

\subsubsection{Other Variation}

\subsection{Stochastic Model}
Also known as epidemiological \gls{abm}, stochastic model is a special use case of a broader class of \gls{abm}. \gls{abm} is a class of computational models to simulate actions and interactions between "agents". The first version of \gls{abm} can be track back to 1971\cite{Schelling1971}. More recently \gls{abm} have been used to inform public health interventions against flu\cite{Ferguson2006a, Ferguson2005} and COVID-19\cite{Maziarz2020, Ferguson2020, Chang2020}

Individual based, simulated version of model, require multiple run to simulate.

\subsection{Forward Step - From Mathematical Formula to Real World}

\subsection{Fitting to Real World Data}
\gls{mcmc} is mostly commonly used method to fit real world data.

\gls{mcmc}

\subsection{Estimating Uncertainty}

\section{Entomology}
\subsection{Endophily}
Research papers often take endophily as a fixed parameter in the model, which means for each different vector there is one according level of endophily.

And the model used in the research often use a 

Does endophily change over time or based on the seasons? Often in the cold weather, some vector species tend to reside in the indoor environment. Not covered in the papers.

TODO:
Create a database from research papers to see the different settings of endophily of vector species

\paragraph{Daily survival rate}

\paragraph{Human visiting rate}

\paragraph{Extrinsic Incubation Period}
Average incubation period in days
Extrinsic incubation period
Features describing a group of vectors(may influenced by interventions)
The composition of the species of given an area
The distribution of vectors
The overall birth rates of vectors
	Types of birth rates defined in the papers
fixed (constant)
The distribution of the length of feeding cycle
Related to species, interventions used
TODO: add two figures(using normal distribution)
Distribution before intervention
Distribution after intervention

The distribution of the life span of a given group

\paragraph{Cycling Repeating Rate}
Cycling repeating rate

\paragraph{Flying ability}
How long could it fly in a given time

Mathematical function:

A spatial distribution for a given time, e.g.

The ability could determine whether an infectious bite could be transmitted from one to another in a long distance.

TODO:
Add distribution example
Add mathematical function
Collect information from papers and create a database
\paragraph{HBI}
HBI - human blood index
	Related to species, intervention
	Usage of LLINs and/or IRS will discourage human biting divert more bites onto non-human hosts

	The birth rates may vary in the different settings. E.g. for a colony of germ in the , the reproduction process follows to a logistic growth:
TODO: added an example figure of logistic growth
	
	Does the birth rate of vectors follow the same line?

\section{Interventions}

Two core interventions defined by WHO: \gls{llins}, \gls{irs}

\subsection{Costs}
Cost of Interventions increases when coverage increase. \cite{Winskill2017a} considered two approaches to simulate the function:
\begin{itemize}
	\item linear increases in cost associated with increasing coverage
	\item delivery cost increases logarithmically with increasing coverage
\end{itemize}

Total cost was assumed to consist of two components: the commodity and the delivery. The commodity part of cost follows U-shape curves follows the marginal cost formula.

$$ MC = \frac{\Delta C}{\Delta Q} $$

In the short run, marginal cost ($MC$) drops when quantity($Q$) increases, then increase as $Q$ increases.

The delivery part increase as coverage increase. For the linear assumption, j

\section{Diseases}
States:
\begin{itemize}
	\item Susceptible
	\item Treating
	\item Disease(with symptom)
	\item Patently Asymptomatic
	\item Sub-patent stage
	\item Prophylactic protection
\end{itemize}
\section{Demography}
\subsection{Age}
Human population data
Often lack in some areas
Could be estimated through satellite observations
Paper high resolution population maps for low income nations: combining land cover and census in East Africa

\subsection{Birth and Death}
\subsection{In and Out}

\section{immunity}

\section{Parameters}

% appendix
\appendix
\printglossaries
\printnomenclature
\bibliographystyle{unsrt}
\bibliography{library}

\end{document}
